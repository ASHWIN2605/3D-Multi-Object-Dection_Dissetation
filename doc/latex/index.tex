Ubuntu 14.\+04

\href{https://travis-ci.org/PRBonn/depth_clustering}{\tt } \href{https://www.codacy.com/project/zabugr/depth_clustering/dashboard?utm_source=github.com&amp;utm_medium=referral&amp;utm_content=PRBonn/depth_clustering&amp;utm_campaign=Badge_Grade_Dashboard}{\tt } \href{https://coveralls.io/github/PRBonn/depth_clustering}{\tt }

This is a fast and robust algorithm to segment point clouds taken with Velodyne sensor into objects. It works with all available Velodyne sensors, i.\+e. 16, 32 and 64 beam ones.

Check out a video that shows all objects outlined in orange\+: \href{https://www.youtube.com/watch?v=UXHX9kFGXfg}{\tt html doc/pics/depth\+\_\+clustering\+\_\+new\+\_\+short\+\_\+1.\+gif \char`\"{}\+Segmentation illustration\char`\"{}}

\subsection*{Prerequisites}

I recommend using a virtual environment in your catkin workspace ({\ttfamily $<$catkin\+\_\+ws$>$} in this readme) and will assume that you have it set up throughout this readme. Please update your commands accordingly if needed. I will be using {\ttfamily pipenv} that you can install with {\ttfamily pip}.

\subsubsection*{Set up workspace and catkin}

Regardless of your system you will need to do the following steps\+: 
\begin{DoxyCode}
cd <catkin\_ws>            # navigate to the workspace
pipenv shell --fancy      # start a virtual environment
pip install catkin-tools  # install catkin-tools for building
mkdir src                 # create src dir if you don't have it already
# Now you just need to clone the repo:
git clone https://github.com/PRBonn/depth\_clustering src/depth\_clustering
\end{DoxyCode}


\subsubsection*{System requirements}

You will need Open\+CV, Q\+G\+L\+Viewer, Free\+G\+L\+UT, Q\+T4 or Q\+T5 and optionally P\+CL and/or R\+OS. The following sections contain an installation command for various Ubuntu systems (click folds to expand)\+:

$<$details$>$

\paragraph*{Install these packages\+:}


\begin{DoxyCode}
sudo apt install libopencv-dev libqglviewer-dev freeglut3-dev libqt4-dev
\end{DoxyCode}
 $<$/details$>$

$<$details$>$ 

Ubuntu 16.\+04

\paragraph*{Install these packages\+:}


\begin{DoxyCode}
sudo apt install libopencv-dev libqglviewer-dev freeglut3-dev libqt5-dev
\end{DoxyCode}
 $<$/details$>$

$<$details$>$ 

Ubuntu 18.\+04

\paragraph*{Install these packages\+:}


\begin{DoxyCode}
sudo apt install libopencv-dev libqglviewer-dev-qt5 freeglut3-dev qtbase5-dev 
\end{DoxyCode}
 $<$/details$>$

\subsubsection*{Optional requirements}

If you want to use P\+CL clouds and/or use R\+OS for data acquisition you can install the following\+:
\begin{DoxyItemize}
\item (optional) P\+CL -\/ needed for saving clouds to disk
\item (optional) R\+OS -\/ needed for subscribing to topics
\end{DoxyItemize}

\subsection*{How to build?}

This is a catkin package. So we assume that the code is in a catkin workspace and C\+Make knows about the existence of Catkin. It should be already taken care of if you followed the instructions \href{#set-up-workspace-and-catkin}{\tt here}. Then you can build it from the project folder\+:


\begin{DoxyCode}
mkdir build
cd build
cmake ..
make -j4
ctest -VV  # run unit tests, optional
\end{DoxyCode}


It can also be built with {\ttfamily catkin\+\_\+tools} if the code is inside catkin workspace\+:


\begin{DoxyCode}
catkin build depth\_clustering
\end{DoxyCode}


P.\+S. in case you don\textquotesingle{}t use {\ttfamily catkin build} you \href{https://catkin-tools.readthedocs.io/en/latest/installing.html}{\tt should} reconsider your decision.

\subsection*{How to run?}

See \href{examples/}{\tt examples}. There are R\+OS nodes as well as standalone binaries. Examples include showing axis oriented bounding boxes around found objects (these start with {\ttfamily show\+\_\+objects\+\_\+} prefix) as well as a node to save all segments to disk. The examples should be easy to tweak for your needs.

\subsection*{Run on real world data}

Go to folder with binaries\+: 
\begin{DoxyCode}
cd <path\_to\_project>/build/devel/lib/depth\_clustering
\end{DoxyCode}


\paragraph*{Frank Moosmann\textquotesingle{}s \char`\"{}\+Velodyne S\+L\+A\+M\char`\"{} Dataset}

Get the data\+: 
\begin{DoxyCode}
mkdir data/; wget http://www.mrt.kit.edu/z/publ/download/velodyneslam/data/scenario1.zip -O
       data/moosmann.zip; unzip data/moosmann.zip -d data/; rm data/moosmann.zip
\end{DoxyCode}


Run a binary to show detected objects\+: 
\begin{DoxyCode}
./show\_objects\_moosmann --path data/scenario1/
\end{DoxyCode}


Alternatively, you can run the data from Qt G\+UI (as in video)\+: 
\begin{DoxyCode}
./qt\_gui\_app
\end{DoxyCode}
 Once the G\+UI is shown, click on {\ttfamily Open\+Folder} button and choose the folder where you have unpacked the {\ttfamily png} files, e.\+g. {\ttfamily data/scenario1/}. Navigate the viewer with arrows and controls seen on screen.

\paragraph*{Other data}

There are also examples on how to run the processing on K\+I\+T\+TI data and on R\+OS input. Follow the {\ttfamily -\/-\/help} output of each of the examples for more details.

Also you can load the data from the G\+UI. Make sure you are loading files with correct extension ({\ttfamily $\ast$.txt} and {\ttfamily $\ast$.bin} for K\+I\+T\+TI, {\ttfamily $\ast$.png} for Moosmann\textquotesingle{}s data).

\subsection*{Documentation}

You should be able to get Doxygen documentation by running\+: 
\begin{DoxyCode}
cd doc/
doxygen Doxyfile.conf
\end{DoxyCode}


\subsection*{Related publications}

Please cite related papers if you use this code\+:


\begin{DoxyCode}
@InProceedings\{bogoslavskyi16iros,
title     = \{Fast Range Image-Based Segmentation of Sparse 3D Laser Scans for Online Operation\},
author    = \{I. Bogoslavskyi and C. Stachniss\},
booktitle = \{Proc. of The International Conference on Intelligent Robots and Systems (IROS)\},
year      = \{2016\},
url       = \{http://www.ipb.uni-bonn.de/pdfs/bogoslavskyi16iros.pdf\}
\}
\end{DoxyCode}



\begin{DoxyCode}
@Article\{bogoslavskyi17pfg,
title   = \{Efficient Online Segmentation for Sparse 3D Laser Scans\},
author  = \{I. Bogoslavskyi and C. Stachniss\},
journal = \{PFG -- Journal of Photogrammetry, Remote Sensing and Geoinformation Science\},
year    = \{2017\},
pages   = \{1--12\},
url     = \{https://link.springer.com/article/10.1007%2Fs41064-016-0003-y\},
\}
\end{DoxyCode}
 